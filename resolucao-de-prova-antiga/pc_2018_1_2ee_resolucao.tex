\documentclass[12pt, a4paper]{article}

%%%%%% 	pacotes	%%%%%%%
\usepackage[utf8]{inputenc}
\usepackage[portuguese]{babel}
\usepackage[T1]{fontenc}
%\usepackage{fontspec} % habilita o comando /setmainfont{times new roman}
\usepackage{graphicx, wrapfig} % para importar imagens e para coloca-las ao lado do texto

\usepackage{amsmath, amsfonts, amssymb}

%\usepackage{blindtext} % para gerar textos com o comando \blindtext[1]

\usepackage[left=3cm,right=2cm,top=3cm,bottom=2cm]{geometry} % definindo as medidas das margens do papel

%%%%%%%	preambulo	%%%%%%
%\setmainfont{Times New Roman}
\setlength{\baselineskip}{1.5cm} %define a distancia/espaçamento entre linhas para ser 1.5 cm (o padrao da norma ABNT)
%\setlength{\parindent}{1.25cm} %define o recuo do paragrafo para ser 1.25cm (o padrao da norma ABNT)
\setlength{\parindent}{0cm}

\begin{document}
\pagestyle{empty}

\begin{wrapfigure}{L}{0.1\textwidth} % o parametro L: a posição da imagem em relação ao texto: a esquerda do texto. outros valores: l, r, i, o, R, O, I. Já o segundo parametro indica o quao perto o texto esta da imagem. Mudar apenas o valor numerico
\includegraphics[width=0.065\textheight]{logo ufpe.jpg}
%separando cada linha por multiplos de 1.5cm, o espaçamento padrão segundo as normas da ABNT.

\end{wrapfigure}

\noindent UNIVERSIDADE FEDERAL DE PERNAMBUCO\\CENTRO DE CIÊNCIAS EXATAS E DA NATUREZA\\DEPARTAMENTO DE MATEMÁTICA\\PRINCÍPIOS DE CONTAGEM - 2023.1\\PROFESSOR: WILLIKAT BEZERRA DE MELO\\TURMA: 2Z\\

\begin{flushleft}

MONITOR: JARDEL FELIPE CABRAL DOS SANTOS\\[1cm] 
\end{flushleft}

\begin{center} \textbf{RESOLUÇÃO DO SEGUNDO EXERCÍCIO ESCOLAR DE 2018.1\\[1cm]}
\end{center}

\textbf{1. De quantos modos 9 amigos podem formar uma roda de ciranda de modo que Abel e Caim não fiquem juntos?} \\
 
Formaremos a roda de ciranda da seguinte maneira: \\

(1) Forme uma roda de ciranda com todos os amigos exceto Abel e Caim. \\

Há \(Q^{7}_{7} = \dfrac{P^{7}_7}{7} = \dfrac{7!}{7} = 6!\) maneiras de formar tal roda de ciranda. \\

(2) Coloque Abel entre duas pessoas da roda formada no item anterior. \\

Como temos \(7\) pessoas na roda, então haverá \(7\) locais em que se pode colocar Abel na roda. (Verifique!) \\

(3) Após colocar Abel na roda, coloque Caim na roda de ciranda formada de modo que ele não fique junto com Abel.  \\ 

Abel está ocupando um dos \(7\) locais mencionados no item (2). Para que Abel e Caim não fiquem juntos, Caim não deve ser colocado entre as mesmas pessoas que Abel. Assim, há \(6\) locais possíveis para colocar Caim na roda de ciranda. \\

Desse modo, pelo princípio multiplicativo, há \(6! \times 7 \times 6 = 30240\) maneiras de formar uma roda de ciranda que respeita as condições impostas no enunciado. \\

\textbf{2. Quantos anagramas possui a palavra TARTARUGA?} \\

TARTARUGA possui \(P_{(9; 2, 3, 2, 1, 1)}=\dfrac{9!}{2! \times 3! \times 2! \times 1! \times 1!} = 15120\) anagramas. \\


\textbf{3. Determine quantas soluções nos inteiros não-negativos possui a equação \[X+Y+Z+W=5\]} 

Podemos representar cada solução como uma lista ordenada de \(4\) números \((X, Y, Z, W)\). Desse modo, \((1,0,2,2)\) representa a solução: \(X = 1\), \(Y=0\), \(Z = W = 2\). As condições que cada lista deve respeitar são: \\ 

(i) \(X, Y, Z, W \in \mathbb{Z}_+\)

(ii) \(X+Y+Z+W = 5\)\\

Dessas duas podemos deduzir que \(0 \leq A  \leq 5, \; \forall A \in \{X,Y,Z,W\}\). Assim, contar quantas soluções nos inteiros não-negativos possui a equação é equivalente a contar quantas listas ordenadas diferentes que respeitam as condições (i) e (ii) existem.\\

Para resolver esse problema de contar listas, será introduzido um novo problema e mostraremos que resolvê-lo é equivalente à resolver o problema da contagem de listas: \\

\textbf{Problema: dadas as letras A,A,A,A,A,B,B,B; quantos anagramas podem ser formados usando todas elas?} \\

A resposta desse problema é \(P_{(8; 5,3)} = \dfrac{8!}{5!\times 3!}\) maneiras. \\

Considere um anagrama qualquer formado com essas letras. Por exemplo: ABAAABAB. Podemos associá-lo a uma única lista \((X,Y,Z,W)\) que satisfaz as propriedades (i) e (ii) mencionadas acima. A associação aconteceria da seguinte maneira: \\

(a) \(X\) seria igual à quantidade de letras ``A'' antes da primeira letra ``B''. \\
(b) \(Y\) seria igual à quantidade de letras ``A'' entre a primeira e a segunda letra ``B''. \\
(c) \(Z\) seria igual à quantidade de letras ``A'' entre a segunda e a terceira letra ``B''. \\
(d) \(W\) seria igual à quantidade de letras ``A'' após a terceira letra ``B''. \\

Observação: se, em uma das situações, não houver letras ``A'', diremos que a quantidade será zero.\\

Assim, a lista \((X,Y,Z,W)\) associada com o anagrama ABAAABAB é \((1,3,1,0)\). Note que a recíproca também é válida: podemos associar cada lista que satisfaz as propriedades (i) e (ii) com um único anagrama formado utilizando \(5\) letras ``A'' e \(3\) letras ``B''. \\

Desse modo, estabelecemos a equivalência entre os dois problemas, pois a quantidade de anagramas será numericamente igual à quantidade de listas. Portanto, quantidade de soluções inteiras não-negativas da equação é: \[P_{(8; 5,3)} = \dfrac{8!}{5!\times 3!} = 56\] 

Nessa fórmula, perceba que \(5\) corresponde ao resultado da soma das variáveis, \(3\) corresponde à quantidade de sinais ``\(+\)'' existentes na equação e \(8\) corresponde à soma de \(5\) e \(3\). Isto será importante para as questões seguintes. \\

\textbf{4. Determine o número de soluções, nos inteiros positivos, da inequação \[X+Y+Z+W<9\]} 

Como foi pedido soluções nos inteiros positivos, temos que \(X, Y,Z,W \in \mathbb{Z}\) e também: \(A \geq 1, \; \forall A \in \{X, Y, Z, W\}\). Desta segunda condição, concluí-se que \[X+Y+Z+W \geq 1+1+1+1=4\]Para que a inequação \(X+Y+Z+W <9\) seja satisfeita, deveremos ter: \\

(i) ou \(X+Y+Z+W = 4\)

(ii) ou \(X+Y+Z+W = 5\)

(iii) ou \(X+Y+Z+W = 6\)

(iv) ou \(X+Y+Z+W = 7\)

(v) ou \(X+Y+Z+W = 8\) \\

Não é difícil ver que nenhum dos casos acima possui solução em comum. Desse modo, após contar o número de soluções inteiras positivas de cada caso, podemos utilizar o princípio aditivo para encontrar o número total de soluções da inequação. \\

Calcularemos, em detalhes, apenas a quantidade de soluções inteiras positivas do caso (ii), visto que os cálculos serão análogos para os demais casos.\\

\textbf{Problema: Determinar a quantidade de soluções inteiras positivas da equação \[X+Y+Z+W = 5\]} 

A única diferença desse problema para o da questão 3 é que agora todas as variáveis (\(X, Y, Z,W\)) devem ser maiores ou iguais a \(1\). Desse modo, não poderemos aplicar diretamente a técnica de contagem vista na questão 3, pois ela fornece soluções inteiras não-negativas. \\

Porém, exigir que uma das variáveis, digamos \(X\), seja inteira e seja maior ou igual a \(1\) é equivalente a exigir que \(X = \overline{X} + 1\), onde \(\overline{X} \in \mathbb{Z}\) e \(\overline{X} \geq 0\). Assim, aplicando essa noção para cada variável na equação, podemos reescrever a equação como: \[(\overline{X}+1)+(\overline{Y}+1)+(\overline{Z}+1)+(\overline{W}+1)=5\]A partir dela, teremos \[\overline{X}+\overline{Y}+\overline{Z}+\overline{W}+4=5 \Longleftrightarrow   \overline{X}+\overline{Y}+\overline{Z}+\overline{W}=1\] 

Podemos associar de maneira biunívoca cada solução inteira não-negativa dessa nova equação com uma solução inteira positiva da equação \(X+Y+Z+W=5\). Desse modo, a quantidade de soluções das equações  serão numericamente iguais. Assim, utilizando a técnica de contagem vista na questão 3 para a nova equação, encontraremos a quantidade de soluções do caso (ii). \\

A equação mencionada no caso (ii) possuirá \[P_{(4;1,3)}=\dfrac{4!}{1!\times 3!} = 4 \text{ soluções.}\]

Daí, não é difícil verificar que o caso \\


(i) possui \(P_{(3;0,3)}=\dfrac{3!}{0!\times 3!} = 1 \text{ solução.}\) \\

(ii) possui \(P_{(4;1,3)}=\dfrac{4!}{1!\times 3!} = 4 \text{ soluções.}\) \\

(iii) possui \(P_{(5;2,3)}=\dfrac{5!}{2!\times 3!} = 10 \text{ soluções.}\) \\

(iv) possui \(P_{(6;3,3)}=\dfrac{6!}{3!\times 3!} = 20 \text{ soluções.}\) \\

(v) possui \(P_{(7;4,3)}=\dfrac{7!}{4!\times 3!} = 35 \text{ soluções.}\) \\

Pelo princípio aditivo, a quantidade de soluções, nos inteiros positivos, da inequação \(X+Y+Z+W<9\) é igual a \(1+4+10+20+35=70\). \\


\textbf{5. O Superhomem necessita comprar 20 capas novas que podem vir nas seguintes cores: encarnado, grená, vermelho e vinho. De quantos modos esta compra pode ser feita caso Superhomem deseje pelo menos 5 capas vermelhas?} \\

Sejam \(E\), \(G\), \(R\) e \(V\) o número de capas compradas de cor encarnado, grená, vermelho e vinho, respectivamente. Como serão compradas \(20\) capas, temos que \[E+G+R+V=20\]

Além disso, precisamos que \(R \geq 5\). Como foi feito na questão 4, podemos garantir esta condição ao considerar \(R = \overline{R} + 5\), onde \(\overline{R} \in \mathbb{Z}\) e \(\overline{R} \geq 0\). Daí, fazendo a substituição apenas da variável \(R\) na equação, teremos:\[E+G+(\overline{R}+5)+V=20 \Longleftrightarrow E+G+\overline{R}+V = 15\]

Há \(P_{(18;15,3)}=\dfrac{18!}{15! \times 3!} = 816\) soluções para esta equação. Logo, há \(816\) modos de realizar as compras, respeitando as condições impostas.


\end{document}