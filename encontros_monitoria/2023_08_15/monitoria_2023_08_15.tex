\documentclass[12pt, a4paper]{article}

%%%%%% 	pacotes	%%%%%%%
\usepackage[utf8]{inputenc}
\usepackage[portuguese]{babel}
\usepackage[T1]{fontenc}
%\usepackage{fontspec} % habilita o comando /setmainfont{times new roman}
\usepackage{graphicx, wrapfig} % para importar imagens e para coloca-las ao lado do texto

\usepackage{amsmath, amsfonts, amssymb}

%\usepackage{blindtext} % para gerar textos com o comando \blindtext[1]

\usepackage[left=3cm,right=2cm,top=3cm,bottom=2cm]{geometry} % definindo as medidas das margens do papel

%%%%%%%	preambulo	%%%%%%
%\setmainfont{Times New Roman}
\setlength{\baselineskip}{1.5cm} %define a distancia/espaçamento entre linhas para ser 1.5 cm (o padrao da norma ABNT)
%\setlength{\parindent}{1.25cm} %define o recuo do paragrafo para ser 1.25cm (o padrao da norma ABNT)
\setlength{\parindent}{0cm}

\begin{document}
\pagestyle{empty}

\begin{wrapfigure}{L}{0.1\textwidth} % o parametro L: a posição da imagem em relação ao texto: a esquerda do texto. outros valores: l, r, i, o, R, O, I. Já o segundo parametro indica o quao perto o texto esta da imagem. Mudar apenas o valor numerico
\includegraphics[width=0.065\textheight]{logo ufpe.jpg}
%separando cada linha por multiplos de 1.5cm, o espaçamento padrão segundo as normas da ABNT.

\end{wrapfigure}

\noindent UNIVERSIDADE FEDERAL DE PERNAMBUCO\\CENTRO DE CIÊNCIAS EXATAS E DA NATUREZA\\DEPARTAMENTO DE MATEMÁTICA\\PRINCÍPIOS DE CONTAGEM - 2023.1\\PROFESSOR: WILLIKAT BEZERRA DE MELO\\TURMA: 2Z\\

\begin{flushleft}

MONITOR: JARDEL FELIPE CABRAL DOS SANTOS\\[1cm] 
\end{flushleft}

\begin{center} \textbf{ENCONTRO DE MONITORIA - 15/08/2023\\[1cm]}
\end{center}

\begin{center}
\textbf{PROBLEMAS}
\end{center}

\textbf{1. De um baralho de \textit{Poker} (7, 8, 9, 10, valete, dama, rei e ás, cada um desses grupos aparecendo em 4 naipes: copas, ouros, paus, espadas), sacam-se simultaneamente 5 cartas. Quantas são as extrações nas quais se forma: \\ \\(a) uma trinca (três cartas em um grupo e as outras duas em dois outros grupos diferentes)? \\ \\(b) um ``flush''  (5 cartas do mesmo naipe, não sendo elas de grupos consecutivos)?} \\

\textbf{2. Quantas rodas de ciranda com exatamente 5 pessoas podem ser formadas de um grupo de 8 pessoas?} 

\begin{center}
\textbf{RESOLUÇÃO}
\end{center}

\textbf{1.}

Note que não existe uma ordem para as cartas pois elas estão sendo sacadas simultaneamente. Assim, \\

(a)

Há \(C^8_1 = 8\) maneiras de escolher o grupo de cartas que formarão a trinca, pois existem \(8\) grupos de cartas. Feita essa escolha, há \(C^4_3 = 4\) maneiras de escolher os naipes das \(3\) cartas.  Resta escolher os grupos e naipes das duas cartas que sobraram. \\

Há \(C^7_2 = 21\) maneiras de escolher os grupos das cartas, visto que elas devem ser de grupos distintos e não podem ser do grupo escolhido para a trinca. Por fim, para cada carta há \(C^4_1 = 4\) maneiras de escolher seu naipe. Não há problema haver duas ou mais cartas de um mesmo naipe. \\

Desse modo, pelo princípio multiplicativo, há \(8\cdot{4}\cdot{21}\cdot{4}\cdot{4} = 12288\) extrações que satisfazem a propriedade desejada. \\

(b)

Há \(C^4_1 = 4\) maneiras de escolher o naipe das cartas que formarão o ``flush''. Feita essa escolha, há \(C^8_5 = 56\) maneiras de escolher os grupos das \(5\) cartas. Porém, dos grupos escolhidos estão incluidos àqueles que são consecutivos. Uma maneira de calcular quantos são é listando todos eles: 

\begin{enumerate}
\item O conjunto \(\{7, 8, 9, 10, J\}\)

\item O conjunto \(\{8, 9, 10, J, Q\}\)

\item O conjunto \(\{9, 10, J, Q, K\}\)

\item O conjunto \(\{10, J, Q, K, A\}\)
\end{enumerate} 

Desse modo, a quantidade de grupos não consecutivos de \(5\) cartas é \(56 - 4 = 52\). Assim, pelo princípio multiplicativo, há \(4\cdot{52}= 208\) extrações que satisfazem a propriedade desejada. \\

\textbf{2.}

Há \(C^8_5 = 56\) maneiras de escolher as pessoas que farão parte da roda de ciranda. Feita essa escolha, há \(Q^5_5 = 4! = 24\) maneiras de formar rodas de ciranda com as pessoas escolhidas. 
Desse modo, pelo princípio multiplicativo, há \(56\cdot{24}= 1344\) rodas de cirandas que podem ser formadas satisfazendo a condição estabelecida. \\


\end{document}