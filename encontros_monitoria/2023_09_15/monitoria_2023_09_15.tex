\documentclass[12pt, a4paper]{article}

%%%%%% 	pacotes	%%%%%%%
%\usepackage[utf8]{inputenc}
\usepackage[portuguese]{babel}
\usepackage[T1]{fontenc}
%\usepackage{fontspec} % habilita o comando /setmainfont{times new roman}
\usepackage{graphicx, wrapfig} % para importar imagens e para coloca-las ao lado do texto
\usepackage{hyperref} % para fazer o sumario ficar interativo e adicionar hiperlinks
\usepackage{amsmath, amsfonts, amssymb, amsthm, mathrsfs} %o penultimo para teoremas e o ultimo para o comando \mathscr{<texto>}

%\usepackage{blindtext} % para gerar textos com o comando \blindtext[1]

\usepackage[left=3cm,right=2cm,top=3cm,bottom=2cm]{geometry} % definindo as medidas das margens do papel


%%%%%%%	preambulo	%%%%%%
%\setmainfont{Times New Roman}
%\setlength{\parskip}{18pt} %18pt/12pt = 1.5, o padrao abnt. define a distancia entre linhas para ser 18pt, como se houvesse uma palavra de tamanho 18pt entre as linhas 

%\setlength{\parindent}{1.25cm} %define o recuo do paragrafo para ser 1.25cm (o padrao da norma ABNT)
\setlength{\parindent}{0cm}

\hypersetup{colorlinks=true, %set true if you want colored links
    linktoc=all,     %set to all if you want both sections and subsections linked
    linkcolor=red,  %choose some color if you want links to stand out
} %https://tex.stackexchange.com/questions/73862/how-can-i-make-a-clickable-table-of-contents


\theoremstyle{definition} \newtheorem{prob}{Problema}
\newtheorem{res}{Resolução do problema}
\theoremstyle{plain} \newtheorem*{teo}{Teorema}
\newtheorem*{lem}{Lema}

\begin{document}
\pagestyle{empty}

\begin{wrapfigure}{L}{0.1\textwidth} % o parametro L: a posição da imagem em relação ao texto: a esquerda do texto. outros valores: l, r, i, o, R, O, I. Já o segundo parametro indica o quao perto o texto esta da imagem. Mudar apenas o valor numerico
\includegraphics[width=0.065\textheight]{logo ufpe.jpg}
%separando cada linha por multiplos de 1.5cm, o espaçamento padrão segundo as normas da ABNT.

\end{wrapfigure}

\noindent UNIVERSIDADE FEDERAL DE PERNAMBUCO\\CENTRO DE CIÊNCIAS EXATAS E DA NATUREZA\\DEPARTAMENTO DE MATEMÁTICA\\PRINCÍPIOS DE CONTAGEM - 2023.1\\PROFESSOR: WILLIKAT BEZERRA DE MELO\\TURMA: 2Z\\

\begin{flushleft}

MONITOR: JARDEL FELIPE CABRAL DOS SANTOS\\[0.75cm] 
\end{flushleft}

\begin{center} \textbf{ENCONTRO DE MONITORIA - 15/09/2023\\[0.75cm]}
\end{center}

\begin{center}
\section*{\normalsize PROBLEMAS\\[0.25cm]} 
\end{center}

\begin{prob}
Quantos vértices tem um grafo completo de 190 arestas?
\end{prob}

\begin{prob}
Mostre que dados dois vértices \(u\) e \(v\), com \(u \neq v\), de um grafo \(G\), existe um caminho ligando \(u\) a \(v\) se
e somente se existe um passeio ligando \(u\) a \(v\).
\end{prob}

\begin{prob}
Explique a diferença entre subgrafo induzido e subgrafo não induzido.
\end{prob}

\newpage

\begin{center}
\section*{\normalsize RESOLUÇÕES\\[0.25cm]}
\end{center}

\begin{res} %questao 1
\end{res}

Vimos no encontro do dia 29/08/2023 que o número de arestas de um grafo completo (\(K_n\)) de \(n\) vértices é \(\binom{n}{2} = \frac{n(n-1)}{2}\). Como o grafo completo mencionado tem \(190\) arestas, então queremos encontrar \(x \in \mathbb{N}\) tal que \(\frac{x(x-1)}{2}=190\). 

\[\dfrac{x(x-1)}{2}=190 \Longleftrightarrow x(x-1) = 380 \Longleftrightarrow x^2 - x = 380 \Longleftrightarrow x^2 -x - 380 = 0 \Longleftrightarrow\]\[\Longleftrightarrow (x-20)(x+19) = 0
 \Longleftrightarrow x = 20 \text{ ou } x = -19\]

Assim, o grafo completo \(K_x\) tem \(20\) vértices. Logo, \(K_x = K_{20}\).



\begin{res} %questao 2
\end{res}

Uma demonstração mais simples pode ser vista na página 39 das \href{https://drive.google.com/file/d/16Gy9vck48p64A-3u1t2-uUVGOVqOlAOg/view}{notas de aula}. \\

Primeiro vamos mostrar que se existe um caminho ligando \(u\) a \(v\) em \(G\), então existe um passeio ligando \(u\) a \(v\): \\

Suponha que existe um caminho ligando \(u\) a \(v\). Por definição, todo caminho é um passeio. Portanto, existe um passeio ligando \(u\) a \(v\). \\

Resta mostrar que se existe um passeio ligando \(u\) a \(v\) em \(G\), então existe um caminho ligando \(u\) a \(v\): \\

Suponha que existe um passeio ligando \(u\) a \(v\). Sem perda de generalidade, suponha que o passeio é descrito pela sequência \((u, v_1, \ldots, v_n,  v)\), onde \(u, v, v_i \in V(G)\) para todo \(i \in \mathbb{N}_n = \{1, 2, \ldots, n\}\). Vamos construir um caminho de \(u\) a \(v\) a partir deste passeio. \\

Considere o conjunto \(V_0 = \{v_1, \ldots, v_n\}\) e defina \[r_0 = \mathrm{min}\{i \in \mathbb{N}_n : v_i = v_j, \quad \text{para algum j tal que }i < j, \quad v_i, v_j \in V_0\}\] \[s_0 = \mathrm{max}\{j \in \mathbb{N}_n: v_{r_0} = v_j, \quad r_0 \neq j, \quad v_{r_0}, v_j \in V_0\}\]

Por construção e pela definição de passeio, a sequência \[A_0=(u, v_1, \ldots, v_{r_0-1}, v_{s_0}, v_{s_0+1}, \ldots, v)\] descreve um passeio de \(u\) à \(v\). \\

Se o passeio descrito por \(A_0\) é um caminho, então a demonstração está concluída. Assim, suponha que o passeio descrito pela sequência \(A_0\) não é um caminho.\\

Daí, existe \(i, j \in \{1, \ldots, r_0-1, s_0, s_0+1, \ldots, n\}\) tal que \(i \neq j\) e \(v_i = v_j\). Por conta da construção de \(A_0\) e pela definição de \(r_0\) e \(s_0\), que temos que \(i,j \notin \{1, \ldots, r_0-1, s_0\}\). Desse modo, \(i,j \in \{s_0+1, s_0+2, \ldots, n\}\). \\

Sejam \(C_1 = \{s_0+1, s_0+2, \ldots, n\}\) e \(V_1 = \{v_{s_0+1}, v_{s_0+2}, \ldots, v_n\}\). Defina \[r_1 = \mathrm{min}\{i \in C_1 : v_i = v_j, \quad \text{para algum j tal que }i < j, \quad v_i, v_j \in V_1\}\] \[s_1 = \mathrm{max}\{j \in C_1: v_{r_1} = v_j, \quad r_1 \neq j, \quad v_{r_1}, v_j \in V_1\}\]

Por construção e pela definição de passeio, a sequência \[A_1=(u, v_1, \ldots, v_{r_0-1}, v_{s_0}, \ldots, v_{r_1 -1}, v_{s_1}, \ldots, v)\]descreve um passeio de \(u\) à \(v\). \\

Se o passeio descrito por \(A_1\) é um caminho, então a demonstração está concluída. Assim, suponha que o passeio descrito pela sequência \(A_1\) não é um caminho. De maneira análoga ao que foi feito na construção de \(A_1\), teremos que: \\

Existe \(i, j \in \{s_1+1, s_1+2, \ldots, n\}\) tal que \(i \neq j\) e \(v_i = v_j\). Denotando por \(C_2\) o conjunto  \(\{s_1+1, s_1+2, \ldots, n\}\) e \(V_2 = \{v_{s_1+1}, v_{s_1+2}, \ldots, v_n\}\). Pode-se definir \[r_2 = \mathrm{min}\{i \in C_2 : v_i = v_j, \quad \text{para algum j tal que }i < j, \quad v_i, v_j \in V_2\}\] \[s_2 = \mathrm{max}\{j \in C_1: v_{r_2} = v_j, \quad r_2 \neq j, \quad v_{r_2}, v_j \in V_1\}\]

Por construção e pela definição de passeio, a sequência \[A_2=(u, v_1, \ldots, v_{r_0-1}, v_{s_0}, \ldots, v_{r_1 -1}, v_{s_1}, \ldots, v_{r_2-1}, v_{s_2}, \ldots, v)\]descreve um passeio de \(u\) à \(v\). \\

Se o passeio descrito por \(A_2\) é um caminho, então a demonstração está concluída. Caso não seja, pode-se repetir o algoritmo uma quantidade finita de vezes até obter um caminho levando de \(u\) a \(v\). \\

De fato, cada vez que construímos um passeio a partir do passeio anterior, obtemos uma nova sequência com pelo menos 1 termo a menos que a sequência anterior, que é finita. \\

Em determinado momento, para algum \(k \geq 2\), teremos uma sequência com todos os termos distintos. Essa sequência descreve, por definição, um caminho de \(u\) até \(v\). Portanto, se existe um passeio de \(u\) a \(v\), então existe um caminho de \(u\) até \(v\).

\begin{flushright}
$\blacksquare$
\end{flushright}

\begin{res} %questao 3
\end{res}

Dado um grafo \(G\) e conjunto \(X \subset V(G)\), dizemos que o subgrafo de \(G\) induzido por \(X\) é o grafo no qual \(V(X) = X \) e \(E(X) = \{(u,v): u, v \in X\}\). Se \(E(X) \varsubsetneq \{(u,v): u, v \in X\}\), então dizemos que o subgrafo é não induzido.

 



\end{document}