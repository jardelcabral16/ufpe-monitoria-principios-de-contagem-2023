\documentclass[12pt, a4paper]{article}

%%%%%% 	pacotes	%%%%%%%
%\usepackage[utf8]{inputenc}
\usepackage[portuguese]{babel}
\usepackage[T1]{fontenc}
%\usepackage{fontspec} % habilita o comando /setmainfont{times new roman}
\usepackage{graphicx, wrapfig} % para importar imagens e para coloca-las ao lado do texto
\usepackage{hyperref} % para fazer o sumario ficar interativo e adicionar hiperlinks
\usepackage{amsmath, amsfonts, amssymb, amsthm, mathrsfs} %o penultimo para teoremas e o ultimo para o comando \mathscr{<texto>}

%\usepackage{blindtext} % para gerar textos com o comando \blindtext[1]

\usepackage[left=3cm,right=2cm,top=3cm,bottom=2cm]{geometry} % definindo as medidas das margens do papel


%%%%%%%	preambulo	%%%%%%
%\setmainfont{Times New Roman}
%\setlength{\parskip}{18pt} %18pt/12pt = 1.5, o padrao abnt. define a distancia entre linhas para ser 18pt, como se houvesse uma palavra de tamanho 18pt entre as linhas 

%\setlength{\parindent}{1.25cm} %define o recuo do paragrafo para ser 1.25cm (o padrao da norma ABNT)
\setlength{\parindent}{0cm}

\hypersetup{colorlinks=true, %set true if you want colored links
    linktoc=all,     %set to all if you want both sections and subsections linked
    linkcolor=red,  %choose some color if you want links to stand out
} %https://tex.stackexchange.com/questions/73862/how-can-i-make-a-clickable-table-of-contents


\theoremstyle{definition} \newtheorem{prob}{Problema}
\newtheorem{res}{Resolução do problema}
\theoremstyle{plain} \newtheorem*{teo}{Teorema}
\newtheorem*{lem}{Lema}

\begin{document}
\pagestyle{empty}

\begin{wrapfigure}{L}{0.1\textwidth} % o parametro L: a posição da imagem em relação ao texto: a esquerda do texto. outros valores: l, r, i, o, R, O, I. Já o segundo parametro indica o quao perto o texto esta da imagem. Mudar apenas o valor numerico
\includegraphics[width=0.065\textheight]{logo ufpe.jpg}
%separando cada linha por multiplos de 1.5cm, o espaçamento padrão segundo as normas da ABNT.

\end{wrapfigure}

\noindent UNIVERSIDADE FEDERAL DE PERNAMBUCO\\CENTRO DE CIÊNCIAS EXATAS E DA NATUREZA\\DEPARTAMENTO DE MATEMÁTICA\\PRINCÍPIOS DE CONTAGEM - 2023.1\\PROFESSOR: WILLIKAT BEZERRA DE MELO\\TURMA: 2Z\\

\begin{flushleft}

MONITOR: JARDEL FELIPE CABRAL DOS SANTOS\\[0.75cm] 
\end{flushleft}

\begin{center} \textbf{ENCONTRO DE MONITORIA - 05/09/2023\\[0.75cm]}
\end{center}

\begin{center}
\section*{\normalsize PROBLEMAS\\[0.25cm]} 
\end{center}

\begin{prob}
63127 candidatos compareceram a uma prova do vestibular (25 questões de múltipla-escolha com 5 alternativas por questão). Considere a afirmação: ``Pelo menos dois candidatos responderam de modo idêntico as \(k\) primeiras questões da prova''. Qual é o maior valor de \(k\) para o qual podemos garantir que a afirmação acima é verdadeira?
\end{prob}

\begin{prob}
Quantos grafos distintos com 7 vértices existem? e com \(n\) vértices? 
\end{prob}

\begin{teo}[Lema dos apertos de mão]
Em um grafo \(G\) com número de arestas igual a \(\mathrm{e}(G)\), temos que \[\sum \limits_{v \in V(G)} d_G(v) = 2\cdot{\mathrm{e}(G)}\]
\end{teo}
\begin{prob}
Demonstre o teorema acima.
\end{prob}



\newpage

\begin{center}
\section*{\normalsize RESOLUÇÕES\\[0.25cm]}
\end{center}

\begin{res} %questao 1
\end{res}

As sentenças abaixo serão garantidamente verdadeiras para os mesmos valores de \(k \leq 25\): \\

(1) De \(63127\) candidatos, pelo menos dois deles responderam de modo idêntico as \(k\) primeiras questões da prova do vestibular. \\

(2) De \(63127\) candidatos, pelo menos dois deles responderam de modo idêntico uma prova de \(k\) questões de múltipla-escolha, com 5 alternativas por questão. \\

Desse modo, pode-se analisar quais valores de \(k\) vão satisfazer uma sentença e aplicar essa conclusão para a outra sentença. \\

Para uma prova com \(k\) questões de múltipla-escolha e 5 alternativas por questão,  existem, pelo princípio multiplicativo, \(\underbrace{5\cdot{5}\cdot{\ldots}\cdot{5}}_{k \text{ vezes}} = 5^k\) gabaritos distintos possíveis. \\

Assumindo que cada candidato só faz a prova uma única vez e não marca mais de uma alternativa por questão, cada candidato produzirá exatamente um gabarito. A quantidade de gabaritos produzidos é numericamente igual à quantidade de candidatos (podendo existir gabaritos iguais produzidos por candidatos diferentes). \\

Podemos representar cada gabarito produzido pelos candidatos  como sendo um pombo e cada gabarito possível como sendo uma casa. Dessa maneira, haverão 63127 pombos e \(5^k\) casas. Se \(5^k < 63127\), então, pelo princípio da casa dos pombos, que existirá uma casa com pelo menos dois pombos. Em outras palavras, dois candidatos que responderam de modo idêntico as \(k\) questões da prova. \\

Essa afirmação sempre será verdadeira desde que \(5^k < 63127\). Como foi pedido o maior valor de \(k\) que garante que a afirmativa é verdadeira, então devemos encontrar o maior valor de \(k\) (um número inteiro) tal que \(5^k < 63127\). Temos que: 

\begin{itemize}
\item \(5^1 = 5 < 63127\)

\item \(5^2 = 25 < 63127\)

\item \(5^3 = 125 < 63127\)

\item \(5^4 = 625 < 63127\)

\item \(5^5 = 3125 < 63127\)

\item \(5^6 = 15625 < 63127\)

\item \(5^7 = 78125  > 63127\)
\end{itemize}

Portanto, o maior valor de \(k\) que garante que a afirmação acima é verdadeira é \(k = 6\).

\begin{res} %questao 2
\end{res}
Um grafo \(G\) é definido de maneira única a partir do número de vértices e de como cada par de vértices do grafo interage (se o par forma uma aresta ou não). Vamos encontrar a quantidade de grafos para o caso geral, com \(n\) vértices. \\

Para um grafo de \(n\) vértices, existem \(\binom{n}{2}= \frac{n(n-1)}{2}\) pares de vértices. Cada um deles pode ser (ou não) uma aresta. Assim, o número de arestas de um grafo é no mínimo zero e no máximo \(\frac{n(n-1)}{2}\), podendo também ser qualquer inteiro entre esses dois valores. \\

Quantos grafos existem com \(n\) vértices e: 
\begin{itemize}
\item 0 arestas? Existe 1 grafo com essa propriedade.

\item 1 aresta? Há \(\binom{n}{2}\) pares de vértices para escolher para ser a aresta do grafo. Desse modo, existem \(\binom{n}{2}\) grafos com essa propriedade. 

\item 2 arestas? Há \(\binom{n}{2}\) pares de vértices para escolher para ser a primeira aresta do grafo. Escolhida a primeira aresta, há \(\binom{n}{2}-1\) pares de vértices para escolher para ser a segunda aresta do grafo. Desse modo, como não existe uma ordem entre as arestas (primeira aresta, segunda aresta e etc.) e pelo princípio multiplicativo, a quantidade de grafos com 2 arestas é numericamente igual a \[\dfrac{\binom{n}{2}\cdot{\left[\binom{n}{2}-1\right]}}{2!}\]

\item 3 arestas? Há \(\binom{n}{2}\) pares de vértices para escolher para ser de a primeira das arestas do grafo. Escolhida a primeira aresta, há \(\binom{n}{2}-1\) pares de vértices para escolher para ser a segunda aresta do grafo. Escolhidas a primeira e segunda arestas, há \(\binom{n}{2}-2\) pares de vértices para escolher para ser a terceira aresta do grafo. Desse modo, como não existe uma ordem entre as arestas (primeira aresta, segunda aresta e etc.) e pelo princípio multiplicativo, a quantidade de grafos com 3 arestas é numericamente igual a \[\dfrac{\binom{n}{2}\cdot{\left[\binom{n}{2}-1\right]}\cdot{\left[\binom{n}{2}-2\right]}}{3!}\] 
\[\vdots\]
\item \(k\) arestas (com \(k < \frac{n(n-1)}{2}\))? Como não existe uma ordem entre as arestas (primeira aresta, segunda aresta e etc.) e pelo princípio multiplicativo, a quantidade de grafos com \(k\) arestas é numericamente igual a \[\dfrac{\binom{n}{2}\cdot{\left[\binom{n}{2}-1\right]}\cdot{\hdots}\cdot{\left[\binom{n}{2}-(k-1)\right]}}{k!}\] 
\[\vdots\]
\item \(\frac{n(n-1)}{2}\) arestas? Existe 1 grafo com essa propriedade (o grafo completo \(K_n\)).
\end{itemize}

Denotando \(\binom{n}{2}=\frac{n(n-1)}{2}\) por \(m\), se o número de arestas for \(k\), com \(0 \leq k \leq \frac{n(n-1)}{2}\), então a quantidade de grafos com essa quantidade de arestas pode ser reescrita como \[P^{m}_{m-k} = (m-k)!\binom{m}{m-k} \qquad \text{onde } P^n_r \text{ é a } r\text{-permutação de }n\]

Portanto, a quantidade de grafos distintos com \(n\) vértices é igual a \[P^m_m + P^m_{m-1} + P^m_{m-2} + \ldots +P^m_{1} + P^m_0 = \sum \limits^{m}_{i = 0} P^m_i \qquad \text{com }m = \binom{n}{2}
\]

Encontramos a resposta para o caso geral. Para o caso particular em que \(n = 7\), teremos a quantidade de grafos sendo \[ \sum \limits^{21}_{i = 0} P^{21}_i = P^{21}_0 + P^{21}_1 + \ldots + P^{21}_{21} \qquad \text{com } m =  \binom{7}{2} = 21\]

\begin{res} %questao 3
\end{res}

Para demonstrar o teorema, utilizaremos o seguinte lema: 
\begin{lem}[Princípio da inclusão-exclusão para \(n\) conjuntos]
Sejam \(A_1, A_2, \ldots, A_n\) subconjuntos de \(\Omega\). Observação: \(|X|\) é a quantidade de elementos do conjunto \(X\). Assim, defina \begin{itemize}
\item \(S = |A_1 \cup A_2 \cup \ldots \cup A_n|\) 

\item \(S_1 = \sum \limits_{i = 1}^{n} |A_i|=|A_1| + |A_2| + \ldots + |A_n|\) 

\item \(S_2 = \sum \limits_{1 \leq i < j \leq n} |A_i \cap A_j|= |A_1 \cap A_2| + |A_1 \cap A_3| + \ldots +|A_{n-1} \cap A_n|\) 

\item \(S_3 = \sum \limits_{1 \leq i < j < k \leq n} |A_i \cap A_j \cap A_k| = |A_1 \cap A_2 \cap A_3| +  \ldots + |A_{n-2} \cap A_{n-1} \cap A_n|\) \\

\(\vdots\)
\item \(S_n = |A_1 \cap A_2 \cap  \ldots \cap A_n|\)

\end{itemize}
Então, \[S = \sum \limits_{i = 1}^{n} (-1)^{i+1}S_i= S_1 - S_2 + S_3 - S_4 + \ldots + (-1)^{n+1}S_n\] 
\end{lem}

\textbf{Entendendo o que o lema diz:}\\

O lema nos permite encontrar a quantidade de elementos da união entre os conjuntos a partir da quantidade de elementos de cada conjunto e de suas interseções. Se \(n = 2\), pelo lema teremos que \(|A_1 \cup A_2| = |A_1| + |A_2| - (|A_1 \cap A_2|)\). Já se \(n = 3\), então \(|A_1 \cup A_2 \cup A_3| = |A_1| + |A_2| + |A_3| - (|A_1 \cap A_2|+|A_1 \cap A_3|+|A_2 \cap A_3|)+|A_1 \cap A_2 \cap A_3|\). \\

Uma demonstração do lema pode ser encontrada no apêndice 1 do livro ``\href{https://drive.google.com/file/d/15_eYJr5kWCfwLeKdheFwPzb-UeNx_9VZ/view?usp=drive_link}{Análise Combinatória e Probabilidade}'' de Morgado, Carvalho, Carvalho e Fernandez.\\

\textbf{A notação que será utilizada:} 
 
\begin{itemize}
\item \(v\): um vértice \(v\) do grafo \(G\).

\item \(V(G)\): o conjunto de vértices do grafo \(G\).

\item \(E(G)\): o conjunto de arestas \(e\) do grafo \(G\).

\item \(e_{ij} = \{v_i, v_j\}\): uma aresta do grafo \(G\) formada pelo par de vértices \(v_i\) e \(v_j\), com \(i \neq j\).

\item \(\mathrm{e}(G)\): o número de arestas que o grafo \(G\) possui.

\item \(|X|\): a quantidade de elementos do conjunto \(X\).

\item \(d_G(v)\): o grau do vértice \(v\) do grafo \(G\). \\

Observação: o grau de um vértice \(v\) é a quantidade de arestas que tem \(v\) como um de seus componentes.
\end{itemize}

\textbf{Começando a demonstração:}\\

Um grafo \(G\) que possui uma quantidade de arestas igual a \(\mathrm{e}(G)\) pode ter um número infinitos de vértices. Isso ocorre quando ele possui um número infinito de vértices de grau zero e um número finito de vértices de grau maior do que zero. Suponha que \(G\) seja um grafo que possui uma quantidade de arestas igual a \(\mathrm{e}(G)\). \\

Defina o conjunto \(V'(G) := \{v \in G: d_G(v) > 0\}\), ou seja, o conjunto de todos os vértices de \(G\) que possuem grau maior do que zero. Como a quantidade de elementos de \(V'(G)\) é finita, suponha que \( |V'(G)|=n \). Podemos denotar cada elemento de \(V'(G)\) com um índice \(i\) tal que \(1 \leq i \leq n\). Assim, sem perda de generalidade, \(V'(G) = \{v_1, v_2, \ldots, v_n\}\). \\

Defina agora os conjuntos: 
\begin{itemize}
\item \(A_1 := \{e \in E(G): v_1 \in e\}\)

\item \(A_2 := \{e \in E(G): v_2 \in e\}\) \\

\(\vdots\)

\item \(A_n := \{e \in E(G): v_n \in e\}\)
\end{itemize} 

Ou seja, \(A_i\) é o conjunto das arestas \(e\) do grafo \(G\) tal que o vértice \(v_i\) é um dos componentes de \(e\). Em outras palavras, \(e_{ix} = \{v_i, v_x\}\) para algum \(i, x \in \{1, 2, \ldots, n\}\). \\

Note que \(A_1 \cup A_2 \cup \ldots \cup A_n = E(G)\), pois todas as arestas de \(G\) são formadas a partir dos vértices de \(V'(G)\). Logo, temos que \[|A_1 \cup A_2 \cup \ldots \cup A_n| = |E(G)| = \mathrm{e}(G)\]

Pelo lema, temos que \[\mathrm{e}(G)=|A_1 \cup A_2 \cup \ldots \cup A_n| = \sum \limits_{i = 1}^{n} (-1)^{i+1}S_i= S_1 - S_2 + S_3 - S_4 + \ldots + (-1)^{n+1}S_n\]

Porém, \(e \in A_i\) e \(e \in A_j\) se, e somente se, \(e = \{v_i, v_j\}\). Desse modo, \(A_i \cap A_j\) tem no máximo 1 elemento pois cada aresta é definida a partir de um par de vértices. Assim, teremos que: ou \(A_i \cap A_j = \varnothing\), ou \(A_i \cap A_j = \{e_{ij}\}\). \\

Em ambas as situações, qualquer interseção formada por três ou mais dos conjuntos \(A_i\), com \(1 \leq i \leq n\), não possuirá elementos, por conta da definição dos conjuntos \(A_i\). Logo, \[S_3 = S_4 = S_5 = \ldots = S_n = 0\] 

Assim, \[\mathrm{e}(G)=|A_1 \cup A_2 \cup \ldots \cup A_n| = S_1 - S_2\]

Da maneira que definimos cada conjunto \(A_i\), teremos que \(|A_i| = d_G(v_i)\) para todo \(i \in \mathbb{N}\) tal que \(1 \leq i \leq n\). Portanto, \[S_1 = \sum \limits_{v \in V'(G)} d_G(v) = \sum \limits_{i = 1}^{n} d_G(v_i)\]

Também, por conta da definição dos conjuntos \(A_i\), teremos que \(S_2 = \mathrm{e}(G)\) pois cada aresta de \(G\) é contada exatamente uma vez em \(S_2\). \\

Logo, teremos \[\mathrm{e}(G)= S_1 - S_2 = \sum \limits_{v \in V'(G)} d_G(v) - \mathrm{e}(G) \;  \Longleftrightarrow \sum \limits_{v \in V'(G)} d_G(v) =\mathrm{e}(G)+ \mathrm{e}(G) =2\cdot{ \mathrm{e}(G)}\]

Por fim, note que \[\sum \limits_{v \in V(G)} d_G(v) =\sum \limits_{v \in V'(G)} d_G(v) \; \; + \sum \limits_{v \in V(G) \backslash V'(G)} d_G(v)\] Por definição de \(V'(G)\), teremos que \(V(G) \backslash V'(G)\) será o conjunto de todos os vértices de \(G\) que possuem grau igual a zero. Assim, \[\sum \limits_{v \in V(G) \backslash V'(G)} d_G(v) = 0\]Portanto, \[\sum \limits_{v \in V(G)} d_G(v) =\sum \limits_{v \in V'(G)} d_G(v) \; \; + \sum \limits_{v \in V(G) \backslash V'(G)} d_G(v) = 2\cdot{\mathrm{e}(G)} + 0\]O que implica que \[\sum \limits_{v \in V(G)} d_G(v) = 2\cdot{\mathrm{e}(G)}\] 
\begin{flushright}
$\blacksquare$
\end{flushright}
\end{document}