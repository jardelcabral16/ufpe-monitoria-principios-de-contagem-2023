\documentclass[12pt, a4paper]{article}

%%%%%% 	pacotes	%%%%%%%
%\usepackage[utf8]{inputenc}
\usepackage[portuguese]{babel}
\usepackage[T1]{fontenc}
%\usepackage{fontspec} % habilita o comando /setmainfont{times new roman}
\usepackage{graphicx, wrapfig} % para importar imagens e para coloca-las ao lado do texto
\usepackage{hyperref} % para fazer o sumario ficar interativo e adicionar hiperlinks
\usepackage{amsmath, amsfonts, amssymb, amsthm, mathrsfs} %o penultimo para teoremas e o ultimo para o comando \mathscr{<texto>}

%\usepackage{blindtext} % para gerar textos com o comando \blindtext[1]

\usepackage[left=3cm,right=2cm,top=3cm,bottom=2cm]{geometry} % definindo as medidas das margens do papel


%%%%%%%	preambulo	%%%%%%
%\setmainfont{Times New Roman}
%\setlength{\parskip}{18pt} %18pt/12pt = 1.5, o padrao abnt. define a distancia entre linhas para ser 18pt, como se houvesse uma palavra de tamanho 18pt entre as linhas 

%\setlength{\parindent}{1.25cm} %define o recuo do paragrafo para ser 1.25cm (o padrao da norma ABNT)
\setlength{\parindent}{0cm}

\hypersetup{colorlinks=true, %set true if you want colored links
    linktoc=all,     %set to all if you want both sections and subsections linked
    linkcolor=red,  %choose some color if you want links to stand out
} %https://tex.stackexchange.com/questions/73862/how-can-i-make-a-clickable-table-of-contents


\theoremstyle{definition} \newtheorem{prob}{Problema}
\newtheorem{res}{Resolução do problema}
\theoremstyle{plain} \newtheorem*{teo}{Teorema}
\newtheorem*{lem}{Lema}

\begin{document}
\pagestyle{empty}

\begin{wrapfigure}{L}{0.1\textwidth} % o parametro L: a posição da imagem em relação ao texto: a esquerda do texto. outros valores: l, r, i, o, R, O, I. Já o segundo parametro indica o quao perto o texto esta da imagem. Mudar apenas o valor numerico
\includegraphics[width=0.065\textheight]{logo ufpe.jpg}
%separando cada linha por multiplos de 1.5cm, o espaçamento padrão segundo as normas da ABNT.

\end{wrapfigure}

\noindent UNIVERSIDADE FEDERAL DE PERNAMBUCO\\CENTRO DE CIÊNCIAS EXATAS E DA NATUREZA\\DEPARTAMENTO DE MATEMÁTICA\\PRINCÍPIOS DE CONTAGEM - 2023.1\\PROFESSOR: WILLIKAT BEZERRA DE MELO\\TURMA: 2Z\\

\begin{flushleft}

MONITOR: JARDEL FELIPE CABRAL DOS SANTOS\\[0.75cm] 
\end{flushleft}

\begin{center} \textbf{ENCONTRO DE MONITORIA - 22/09/2023\\[0.75cm]}
\end{center}

\begin{center}
\section*{\normalsize PROBLEMAS\\[0.25cm]} 
\end{center}

\begin{prob}
Demonstre que uma árvore com \(n\) vértices tem no mínimo \(n-1\) vértices.
\end{prob}

\begin{prob}
Quantas folhas uma árvore com 7 vértices pode ter no máximo?
\end{prob}

\newpage

\begin{center}
\section*{\normalsize RESOLUÇÕES\\[0.25cm]}
\end{center}

\begin{res} %questao 1
\end{res}

Para demonstrar esta afirmação, precisaremos de dois teoremas: \\

\textbf{Teorema 1:} Se \(G\) é um grafo conexo com \(n\) vértices, então \(\mathrm{e}(G) \geq n-1\). \\

\textbf{Teorema 2:} Seja \(G\) um grafo conexo com \(n\) vértices. Se \(\mathrm{e}(G) = n-1\), então \(G\) é acíclico. \\

Estes teoremas aparecem nas \href{https://drive.google.com/file/d/16Gy9vck48p64A-3u1t2-uUVGOVqOlAOg/view}{notas de aula} e são demonstrados na página 41 e 42 da mesma. Ver lema 3 e proposição 8. \\

\begin{proof}
Faremos a demonstração em duas etapas: \\

(1) Mostraremos a existência de uma árvore com \(n\) vértices e com \(n-1\) arestas.\\

(2) Argumentaremos que não existe árvores com \(n\) vértices e com menos de \(n-1\) arestas.\\

Seja \(G\) um grafo conexo com \(n\) vértices. Pelo teorema 1, temos que \(\mathrm{e}(G) \geq n-1\). Ou seja, o número de arestas de \(G\) é pelo menos \(n-1\). Para ser uma árvore, \(G\) precisa ser acíclico. Suponha que \(\mathrm{e}(G) = n-1\) (o teorema 1 garante que isso é possível). Logo, pelo teorema 2, temos que \(G\) é acíclico. Portanto, por definição, \(G\) é uma árvore. Desse modo, existe uma árvore com \(n\) vértices e com \(n-1\) arestas. \\

Como toda árvore é um grafo conexo, então, dada uma árvore \(A\) com \(n\) vértices, pelo teorema 1, temos que \(\mathrm{e}(A) \geq n-1\). Logo, o número mínimo de arestas que uma árvore com \(n\) vértices pode ter é \(n-1\) arestas.
  
\end{proof}



\begin{res} %questao 2
\end{res}
Observe na figura a abaixo uma árvore com 7 vértices e 6 folhas: \\

\begin{center}
\includegraphics[scale=0.25]{arvore_6folhas.png}
\end{center}

\textbf{Afirmação:} Uma árvore com 7 vértices tem no máximo 6 folhas. \\


Para demonstrar esta afirmação, utilizaremos o seguinte lema: \\

\textbf{Lema:} Um grafo \(G\) é conexo se e somente se para quaisquer \(x,y \in \mathrm{V}(G)\), existe um caminho de \(x\) a \(y\). \\

Este lema aparece nas \href{https://drive.google.com/file/d/16Gy9vck48p64A-3u1t2-uUVGOVqOlAOg/view}{notas de aula} e é demonstrado na página 40 da mesma. Ver lema 2.

\begin{proof}[Demonstração da afirmação]
Suponha, para efeito de absurdo, que exista uma árvore \(G\) com 7 vértices e com mais de 6 folhas. Pela definição de folha, o número de vértices de grau 1 é igual à quantidade de folhas da árvore. Logo, \(G\) deve ter 7 folhas pois só possui 7 vértices. \\

Considere \(u,v,w \in \mathrm{V}(G)\) tais que \(u\neq v\), \(u \neq w\) e \(v \neq w\). Por ser uma árvore, temos que \(G\) é conexo. Desse modo, pelo lema, para quaisquer \(x,y \in \mathrm{V}(G)\), existe um caminho de \(x\) a \(y\).  Sejam \(A\) e \(B\) sequências de vértices de \(G\) que descrevem um caminho de \(u\) a \(v\) e \(u\) a \(w\), respectivamente. Temos duas possibilidades: \\

(i) ou a sequência \(A\) tem exatamente dois termos. (\(A = (u,v)\)) \\

(ii) ou a sequência \(A\) tem mais de dois termos.  \\

Note que (ii) não pode acontecer, pois implicaria que existe um vértice de \(G\) que tem grau maior do que 1 (esse vértice seria uma ``ponte'' que ligaria dois vértices no caminho) e contradiria a hipótese de \(G\) ter 7 folhas. Assim, necessariamente (i) deve acontecer. \\

De maneira análoga, podemos concluir que \(B\) é uma sequência com exatamente dois termos, ou seja: \(B =(u,w)\). Porém, isso significa que existem as arestas \(e_1 = \{u,v\}\) e \(e_2 =\{u,w\}\) na árvore \(G\) (caso contrário os caminhos entre os vértices não existiriam). Daí, pela definição de grau de um vértice, temos que o grau do vértice \(u\) é pelo menos 2. Desse modo, \(u\) não é uma folha de \(G\). Absurdo! pois supomos que todo vértice de \(G\) é uma folha.\\

O absurdo aconteceu pois supomos a existência das sequências de vértices \(A\) e \(B\). Logo, conclui-se que não existem tais sequências e, portanto, não existem caminhos de \(u\) a \(v\) e de \(u\) a \(w\). Assim, pelo lema, \(G\) não é conexo. Absurdo! pois supomos que \(G\) é uma árvore e toda árvore é conexa. \\

Conclui-se que tal árvore \(G\) não pode existir. Ou seja, não existe uma árvore com 7 vértices e mais de 6 folhas.

\end{proof}
 



\end{document}